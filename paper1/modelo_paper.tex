\documentclass[]{llncs}

\usepackage{hyperref}
\usepackage[pdftex]{color,graphicx}
\usepackage{amssymb}
%\usepackage{amsmath} % assumes amsmath package installed
\usepackage{fancyvrb,fancyhdr}
\usepackage[utf8]{inputenc}
\usepackage{listings}
\usepackage{todonotes}


%*********************** FORMATO DE LISTADOS **************************
\lstset{
  basicstyle=\tiny,
  keywordstyle=\bfseries,
  % underlined bold black keywords
  identifierstyle=, % nothing happens
%  commentstyle=\color{white}, % white comments
  stringstyle=\ttfamily, % typewriter type for strings
  showstringspaces=false,
  frame=single,
	language=Java,
	numbers=left,
	numberstyle=\tiny,
	stepnumber=2,
	numbersep=5pt
}


\author{Gonzalo A. Aranda Corral}
\title{Modelo de paper LLNCS}


\begin{document}
\maketitle
\begin{abstract}
Resumen de la cosa
\end{abstract}

\section{Introduction}
\subsection{Encuadre en la situación mundial}
\subsection{Planteamiento del problema}
\subsection{Soluciones existentes y fallos}
\subsection{Tu propuesta y como corrige fallos}

\section{Desarrollo de tu solución}
\subsection{Ingeniería - Tecnología}
\subsection{Arquitectura de la solución}
\subsection{Qué y Cómo...}

\section{Experimentación y resultados}
\subsection{cubrir casos de uso... o experimentos interesantes}

\section{Conclusiones y trabajos futuros}

\begin{thebibliography}{99}
\bibitem Esto es lo que hay
\end{thebibliography}

\end{document}
